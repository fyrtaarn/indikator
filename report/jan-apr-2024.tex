% Options for packages loaded elsewhere
\PassOptionsToPackage{unicode}{hyperref}
\PassOptionsToPackage{hyphens}{url}
\PassOptionsToPackage{dvipsnames,svgnames,x11names}{xcolor}
%
\documentclass[
  letterpaper,
  DIV=11,
  numbers=noendperiod]{scrartcl}

\usepackage{amsmath,amssymb}
\usepackage{iftex}
\ifPDFTeX
  \usepackage[T1]{fontenc}
  \usepackage[utf8]{inputenc}
  \usepackage{textcomp} % provide euro and other symbols
\else % if luatex or xetex
  \usepackage{unicode-math}
  \defaultfontfeatures{Scale=MatchLowercase}
  \defaultfontfeatures[\rmfamily]{Ligatures=TeX,Scale=1}
\fi
\usepackage{lmodern}
\ifPDFTeX\else  
    % xetex/luatex font selection
\fi
% Use upquote if available, for straight quotes in verbatim environments
\IfFileExists{upquote.sty}{\usepackage{upquote}}{}
\IfFileExists{microtype.sty}{% use microtype if available
  \usepackage[]{microtype}
  \UseMicrotypeSet[protrusion]{basicmath} % disable protrusion for tt fonts
}{}
\makeatletter
\@ifundefined{KOMAClassName}{% if non-KOMA class
  \IfFileExists{parskip.sty}{%
    \usepackage{parskip}
  }{% else
    \setlength{\parindent}{0pt}
    \setlength{\parskip}{6pt plus 2pt minus 1pt}}
}{% if KOMA class
  \KOMAoptions{parskip=half}}
\makeatother
\usepackage{xcolor}
\setlength{\emergencystretch}{3em} % prevent overfull lines
\setcounter{secnumdepth}{-\maxdimen} % remove section numbering
% Make \paragraph and \subparagraph free-standing
\makeatletter
\ifx\paragraph\undefined\else
  \let\oldparagraph\paragraph
  \renewcommand{\paragraph}{
    \@ifstar
      \xxxParagraphStar
      \xxxParagraphNoStar
  }
  \newcommand{\xxxParagraphStar}[1]{\oldparagraph*{#1}\mbox{}}
  \newcommand{\xxxParagraphNoStar}[1]{\oldparagraph{#1}\mbox{}}
\fi
\ifx\subparagraph\undefined\else
  \let\oldsubparagraph\subparagraph
  \renewcommand{\subparagraph}{
    \@ifstar
      \xxxSubParagraphStar
      \xxxSubParagraphNoStar
  }
  \newcommand{\xxxSubParagraphStar}[1]{\oldsubparagraph*{#1}\mbox{}}
  \newcommand{\xxxSubParagraphNoStar}[1]{\oldsubparagraph{#1}\mbox{}}
\fi
\makeatother


\providecommand{\tightlist}{%
  \setlength{\itemsep}{0pt}\setlength{\parskip}{0pt}}\usepackage{longtable,booktabs,array}
\usepackage{calc} % for calculating minipage widths
% Correct order of tables after \paragraph or \subparagraph
\usepackage{etoolbox}
\makeatletter
\patchcmd\longtable{\par}{\if@noskipsec\mbox{}\fi\par}{}{}
\makeatother
% Allow footnotes in longtable head/foot
\IfFileExists{footnotehyper.sty}{\usepackage{footnotehyper}}{\usepackage{footnote}}
\makesavenoteenv{longtable}
\usepackage{graphicx}
\makeatletter
\def\maxwidth{\ifdim\Gin@nat@width>\linewidth\linewidth\else\Gin@nat@width\fi}
\def\maxheight{\ifdim\Gin@nat@height>\textheight\textheight\else\Gin@nat@height\fi}
\makeatother
% Scale images if necessary, so that they will not overflow the page
% margins by default, and it is still possible to overwrite the defaults
% using explicit options in \includegraphics[width, height, ...]{}
\setkeys{Gin}{width=\maxwidth,height=\maxheight,keepaspectratio}
% Set default figure placement to htbp
\makeatletter
\def\fps@figure{htbp}
\makeatother

\usepackage{booktabs}
\usepackage{caption}
\usepackage{longtable}
\usepackage{colortbl}
\usepackage{array}
\usepackage{anyfontsize}
\usepackage{multirow}
\KOMAoption{captions}{tableheading}
\makeatletter
\@ifpackageloaded{caption}{}{\usepackage{caption}}
\AtBeginDocument{%
\ifdefined\contentsname
  \renewcommand*\contentsname{Table of contents}
\else
  \newcommand\contentsname{Table of contents}
\fi
\ifdefined\listfigurename
  \renewcommand*\listfigurename{List of Figures}
\else
  \newcommand\listfigurename{List of Figures}
\fi
\ifdefined\listtablename
  \renewcommand*\listtablename{List of Tables}
\else
  \newcommand\listtablename{List of Tables}
\fi
\ifdefined\figurename
  \renewcommand*\figurename{Figur}
\else
  \newcommand\figurename{Figur}
\fi
\ifdefined\tablename
  \renewcommand*\tablename{Tabell}
\else
  \newcommand\tablename{Tabell}
\fi
}
\@ifpackageloaded{float}{}{\usepackage{float}}
\floatstyle{ruled}
\@ifundefined{c@chapter}{\newfloat{codelisting}{h}{lop}}{\newfloat{codelisting}{h}{lop}[chapter]}
\floatname{codelisting}{Listing}
\newcommand*\listoflistings{\listof{codelisting}{List of Listings}}
\makeatother
\makeatletter
\makeatother
\makeatletter
\@ifpackageloaded{caption}{}{\usepackage{caption}}
\@ifpackageloaded{subcaption}{}{\usepackage{subcaption}}
\makeatother

\ifLuaTeX
\usepackage[bidi=basic]{babel}
\else
\usepackage[bidi=default]{babel}
\fi
\babelprovide[main,import]{norsk}
% get rid of language-specific shorthands (see #6817):
\let\LanguageShortHands\languageshorthands
\def\languageshorthands#1{}
\ifLuaTeX
  \usepackage{selnolig}  % disable illegal ligatures
\fi
\usepackage{bookmark}

\IfFileExists{xurl.sty}{\usepackage{xurl}}{} % add URL line breaks if available
\urlstyle{same} % disable monospaced font for URLs
\hypersetup{
  pdftitle={Trafikkulykker statistikk for første tertial 2024},
  pdflang={no},
  colorlinks=true,
  linkcolor={blue},
  filecolor={Maroon},
  citecolor={Blue},
  urlcolor={Blue},
  pdfcreator={LaTeX via pandoc}}


\title{Trafikkulykker statistikk for første tertial 2024}
\usepackage{etoolbox}
\makeatletter
\providecommand{\subtitle}[1]{% add subtitle to \maketitle
  \apptocmd{\@title}{\par {\large #1 \par}}{}{}
}
\makeatother
\subtitle{01 oktober 2024}
\author{}
\date{}

\begin{document}
\maketitle


Denne rapporten omfatter data fra Felles minimum datasett (FMDS) i Norsk
pasientregister (NPR), hentet fra deltakende helse institusjoner i
Fyrtårnprosjektet for perioden januar til april 2024. Totalt er 43 247
pasienter registert, hvorav 1 259 pasienter hadde mer enn én
skadehendelse. Pasienter med ugyldig personnummer er ekskludert (N =
736). Videre analyser gjelder bare for pasienter involvert i
trafikkulykker (N = 1 047).

Det er en vis usikkerhet med tallene som blir presentert her fordi noen
pasienter med samme skadde kan bli telt flere ganger dersom de er
registrert ved ulike sykehus eller legevakter. Vi jobber fortsatt med å
utvikle metoder for å minimere slike tilfeller.

\section{Alvorlighetsgrad}\label{alvorlighetsgrad}

Alvorlighetsgrad av skade er beskrevet i registreringsveilederen for
personskader (FMDS), som gir en overordnet vurdering av hvor alvorlig
pasienten er skadet, klassifisert for enkelhets skyld ut fra trussel mot
livets opprettholdelse. Inndelingen er basert på en internasjonal
klassifikasjon \emph{Abbreviated Injury Scale (AIS)}.

Fordelingen av alvorlighetsgrad for ulykkeskade pasienter vises i
Tabell~\ref{tbl-alvorlig} nedenfor.

\begingroup
\fontsize{12.0pt}{14.4pt}\selectfont

\begin{longtable}{lrr}

\caption{\label{tbl-alvorlig}Alvorlighetsgrad skade}

\tabularnewline

\toprule
  & N & Prosent \\ 
\midrule\addlinespace[2.5pt]
Liten skade (AIS 1) & 693 & 66.2 \\ 
Moderat skade (AIS 2) & 267 & 25.5 \\ 
Alvorlig skade (AIS 3+) & 64 & 6.1 \\ 
Ukjent alvorlighetsgrad & 23 & 2.2 \\ 
Total & 1047 & 100.0 \\ 
\bottomrule

\end{longtable}

\endgroup

Fordelingen av lettere skadde pasienter med AIS 1 eller 2 (N = 960) og
hardt skadde pasienter med AIS 3+ (N = 64) etter kjønn, aldersgrupper og
fremkomstmiddel vises i Tabell~\ref{tbl-ais} nedenfor.

\begingroup
\fontsize{12.0pt}{14.4pt}\selectfont

\begin{longtable}{lrrrr}

\caption{\label{tbl-ais}Trafikkulykker}

\tabularnewline

\toprule
 & \multicolumn{2}{c}{\textbf{AIS 1 \& 2}} & \multicolumn{2}{c}{\textbf{AIS 3+}} \\ 
\cmidrule(lr){2-3} \cmidrule(lr){4-5}
Beskrivelse & N & Prosent & N & Prosent \\ 
\midrule\addlinespace[2.5pt]
\multicolumn{5}{l}{Kjønn} \\[2.5pt] 
\midrule\addlinespace[2.5pt]
Kvinne & 395 & 41.1 & 26 & 40.6 \\ 
Mann & 565 & 58.9 & 38 & 59.4 \\ 
\midrule\addlinespace[2.5pt]
\multicolumn{5}{l}{Alder} \\[2.5pt] 
\midrule\addlinespace[2.5pt]
0-14 & 92 & 9.6 & 1 & 1.6 \\ 
15-24 & 162 & 16.9 & 10 & 15.6 \\ 
25-44 & 344 & 35.8 & 13 & 20.3 \\ 
45-64 & 253 & 26.4 & 26 & 40.6 \\ 
65-79 & 82 & 8.5 & 10 & 15.6 \\ 
80+ & 27 & 2.8 & 4 & 6.2 \\ 
\midrule\addlinespace[2.5pt]
\multicolumn{5}{l}{Fremkomstmiddel} \\[2.5pt] 
\midrule\addlinespace[2.5pt]
ATV/Firhjuling & 1 & 0.1 & 2 & 3.1 \\ 
Annet spesifisert & 16 & 1.7 & - & - \\ 
Buss & 32 & 3.3 & 1 & 1.6 \\ 
El-sykkel & 21 & 2.2 & - & - \\ 
Elektrisk sparkesykkel & 61 & 6.4 & 4 & 6.2 \\ 
Lastebil & 4 & 0.4 & 1 & 1.6 \\ 
Moped & 22 & 2.3 & - & - \\ 
Motorsykkel & 41 & 4.3 & 4 & 6.2 \\ 
Personbil/Varebil & 294 & 30.6 & 19 & 29.7 \\ 
Ski/Snøbrett mv. & 3 & 0.3 & - & - \\ 
Snøskuter & 3 & 0.3 & - & - \\ 
Sparkesykkel & 4 & 0.4 & - & - \\ 
Sykkel & 191 & 19.9 & 15 & 23.4 \\ 
Til fots & 147 & 15.3 & 10 & 15.6 \\ 
Trikk/tog/bane & 6 & 0.6 & 1 & 1.6 \\ 
Ukjent & 69 & 7.2 & 4 & 6.2 \\ 
\bottomrule

\end{longtable}

\endgroup

Tabell~\ref{tbl-ais} viser at flere menn enn kvinner er skadet i
trafikkulykker, både når det gjelder alvorlige (59.4\%) og mindre
alvorlige skader (58.9\%). Når det gjelder aldersfordeling, ser vi at
antallet hardt skadde er høyest blant personer i aldersgruppen 45-64 år
(40.6\%), mens de med lettere skader er mest representert i
aldersgruppen 25-44 år (35.8\%). Fordelingen av fremkomstmidler ved
trafikkulykker er relativt lik for både hardt og lettere skadde.




\end{document}
