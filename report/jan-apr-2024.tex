% Options for packages loaded elsewhere
\PassOptionsToPackage{unicode}{hyperref}
\PassOptionsToPackage{hyphens}{url}
\PassOptionsToPackage{dvipsnames,svgnames,x11names}{xcolor}
%
\documentclass[
  letterpaper,
  DIV=11,
  numbers=noendperiod]{scrartcl}

\usepackage{amsmath,amssymb}
\usepackage{iftex}
\ifPDFTeX
  \usepackage[T1]{fontenc}
  \usepackage[utf8]{inputenc}
  \usepackage{textcomp} % provide euro and other symbols
\else % if luatex or xetex
  \usepackage{unicode-math}
  \defaultfontfeatures{Scale=MatchLowercase}
  \defaultfontfeatures[\rmfamily]{Ligatures=TeX,Scale=1}
\fi
\usepackage{lmodern}
\ifPDFTeX\else  
    % xetex/luatex font selection
\fi
% Use upquote if available, for straight quotes in verbatim environments
\IfFileExists{upquote.sty}{\usepackage{upquote}}{}
\IfFileExists{microtype.sty}{% use microtype if available
  \usepackage[]{microtype}
  \UseMicrotypeSet[protrusion]{basicmath} % disable protrusion for tt fonts
}{}
\makeatletter
\@ifundefined{KOMAClassName}{% if non-KOMA class
  \IfFileExists{parskip.sty}{%
    \usepackage{parskip}
  }{% else
    \setlength{\parindent}{0pt}
    \setlength{\parskip}{6pt plus 2pt minus 1pt}}
}{% if KOMA class
  \KOMAoptions{parskip=half}}
\makeatother
\usepackage{xcolor}
\setlength{\emergencystretch}{3em} % prevent overfull lines
\setcounter{secnumdepth}{-\maxdimen} % remove section numbering
% Make \paragraph and \subparagraph free-standing
\makeatletter
\ifx\paragraph\undefined\else
  \let\oldparagraph\paragraph
  \renewcommand{\paragraph}{
    \@ifstar
      \xxxParagraphStar
      \xxxParagraphNoStar
  }
  \newcommand{\xxxParagraphStar}[1]{\oldparagraph*{#1}\mbox{}}
  \newcommand{\xxxParagraphNoStar}[1]{\oldparagraph{#1}\mbox{}}
\fi
\ifx\subparagraph\undefined\else
  \let\oldsubparagraph\subparagraph
  \renewcommand{\subparagraph}{
    \@ifstar
      \xxxSubParagraphStar
      \xxxSubParagraphNoStar
  }
  \newcommand{\xxxSubParagraphStar}[1]{\oldsubparagraph*{#1}\mbox{}}
  \newcommand{\xxxSubParagraphNoStar}[1]{\oldsubparagraph{#1}\mbox{}}
\fi
\makeatother


\providecommand{\tightlist}{%
  \setlength{\itemsep}{0pt}\setlength{\parskip}{0pt}}\usepackage{longtable,booktabs,array}
\usepackage{calc} % for calculating minipage widths
% Correct order of tables after \paragraph or \subparagraph
\usepackage{etoolbox}
\makeatletter
\patchcmd\longtable{\par}{\if@noskipsec\mbox{}\fi\par}{}{}
\makeatother
% Allow footnotes in longtable head/foot
\IfFileExists{footnotehyper.sty}{\usepackage{footnotehyper}}{\usepackage{footnote}}
\makesavenoteenv{longtable}
\usepackage{graphicx}
\makeatletter
\def\maxwidth{\ifdim\Gin@nat@width>\linewidth\linewidth\else\Gin@nat@width\fi}
\def\maxheight{\ifdim\Gin@nat@height>\textheight\textheight\else\Gin@nat@height\fi}
\makeatother
% Scale images if necessary, so that they will not overflow the page
% margins by default, and it is still possible to overwrite the defaults
% using explicit options in \includegraphics[width, height, ...]{}
\setkeys{Gin}{width=\maxwidth,height=\maxheight,keepaspectratio}
% Set default figure placement to htbp
\makeatletter
\def\fps@figure{htbp}
\makeatother

\usepackage{booktabs}
\usepackage{caption}
\usepackage{longtable}
\usepackage{colortbl}
\usepackage{array}
\usepackage{anyfontsize}
\usepackage{multirow}
\KOMAoption{captions}{tablesignature}
\makeatletter
\@ifpackageloaded{caption}{}{\usepackage{caption}}
\AtBeginDocument{%
\ifdefined\contentsname
  \renewcommand*\contentsname{Table of contents}
\else
  \newcommand\contentsname{Table of contents}
\fi
\ifdefined\listfigurename
  \renewcommand*\listfigurename{List of Figures}
\else
  \newcommand\listfigurename{List of Figures}
\fi
\ifdefined\listtablename
  \renewcommand*\listtablename{List of Tables}
\else
  \newcommand\listtablename{List of Tables}
\fi
\ifdefined\figurename
  \renewcommand*\figurename{Figur}
\else
  \newcommand\figurename{Figur}
\fi
\ifdefined\tablename
  \renewcommand*\tablename{Tabell}
\else
  \newcommand\tablename{Tabell}
\fi
}
\@ifpackageloaded{float}{}{\usepackage{float}}
\floatstyle{ruled}
\@ifundefined{c@chapter}{\newfloat{codelisting}{h}{lop}}{\newfloat{codelisting}{h}{lop}[chapter]}
\floatname{codelisting}{Listing}
\newcommand*\listoflistings{\listof{codelisting}{List of Listings}}
\makeatother
\makeatletter
\makeatother
\makeatletter
\@ifpackageloaded{caption}{}{\usepackage{caption}}
\@ifpackageloaded{subcaption}{}{\usepackage{subcaption}}
\makeatother

\ifLuaTeX
\usepackage[bidi=basic]{babel}
\else
\usepackage[bidi=default]{babel}
\fi
\babelprovide[main,import]{norsk}
% get rid of language-specific shorthands (see #6817):
\let\LanguageShortHands\languageshorthands
\def\languageshorthands#1{}
\ifLuaTeX
  \usepackage{selnolig}  % disable illegal ligatures
\fi
\usepackage{bookmark}

\IfFileExists{xurl.sty}{\usepackage{xurl}}{} % add URL line breaks if available
\urlstyle{same} % disable monospaced font for URLs
\hypersetup{
  pdftitle={FMDS første tertial 2024},
  pdflang={no},
  colorlinks=true,
  linkcolor={blue},
  filecolor={Maroon},
  citecolor={Blue},
  urlcolor={Blue},
  pdfcreator={LaTeX via pandoc}}


\title{FMDS første tertial 2024}
\usepackage{etoolbox}
\makeatletter
\providecommand{\subtitle}[1]{% add subtitle to \maketitle
  \apptocmd{\@title}{\par {\large #1 \par}}{}{}
}
\makeatother
\subtitle{26 september 2024}
\author{}
\date{}

\begin{document}
\maketitle


Denne rapporten omfatter data fra Felles minimum datasett (FMDS), hentet
fra Norsk pasientregister (NPR) for perioden januar til april 2024.
Totalt er 43 247 pasienter ble registeret, hvorav 1 259 pasienter hadde
mer enn én skadehendelse. Pasienter med ugyldig personnummer er
ekskludert (N = 736). Videre analyser gjelder bare for pasienter
involvert i trafikkulykker (N = 1 047).

\section{Alvorlighetsgrad}\label{alvorlighetsgrad}

Alvorlighetsgrad av skade er beskrevet i registreringsveilederen for
personskader (FMDS), som gir en overordnet vurdering av hvor alvorlig
pasienten er skadet, klassifisert for enkelhets skyld ut fra trussel mot
livets opprettholdelse. Inndelingen er basert på en internasjonal
klassifikasjon \emph{Abbreviated Injury Scale (AIS)}.

Fordelingen av alvorlighetsgrad for ulykkeskade pasienter vises i
Tabell~\ref{tbl-alvorlig} nedenfor.

\begingroup
\fontsize{12.0pt}{14.4pt}\selectfont

\begin{longtable}{lrr}


\caption*{
{\large Alvorlighetsgrad skade}
} \\ 
\toprule
beskrivelse & N & prosent \\ 
\midrule\addlinespace[2.5pt]
Liten skade (AIS 1) & 693 & 66.2 \\ 
Moderat skade (AIS 2) & 267 & 25.5 \\ 
Alvorlig skade (AIS 3+) & 64 & 6.1 \\ 
Ukjent alvorlighetsgrad & 23 & 2.2 \\ 
Total & 1047 & 100.0 \\ 
\bottomrule

\caption{\label{tbl-alvorlig}Alvorlighetsgrad}

\tabularnewline
\end{longtable}

\endgroup

\subsection{Hardt skadd AIS 3+}\label{hardt-skadd-ais-3}

Fordelingen av hardt skadde pasienter med AIS 3+ (N = 64) etter kjønn,
aldersgrupper og fremkomstmiddel er som følger:

\begingroup
\fontsize{12.0pt}{14.4pt}\selectfont
\begin{longtable*}{lrr}
\caption*{
{\large Kjønn fordeling}
} \\ 
\toprule
beskrivelse & N & prosent \\ 
\midrule\addlinespace[2.5pt]
Mann & 38 & 59.4 \\ 
Kvinne & 26 & 40.6 \\ 
Total & 64 & 100.0 \\ 
\bottomrule
\end{longtable*}
\endgroup

\begingroup
\fontsize{12.0pt}{14.4pt}\selectfont
\begin{longtable*}{lrr}
\caption*{
{\large Aldersgrupper}
} \\ 
\toprule
Alder & N & prosent \\ 
\midrule\addlinespace[2.5pt]
0-14 & 1 & 1.6 \\ 
15-24 & 10 & 15.6 \\ 
25-44 & 13 & 20.3 \\ 
45-64 & 26 & 40.6 \\ 
65-79 & 10 & 15.6 \\ 
80+ & 4 & 6.2 \\ 
Total & 64 & 100.0 \\ 
\bottomrule
\end{longtable*}
\endgroup

\begingroup
\fontsize{12.0pt}{14.4pt}\selectfont
\begin{longtable*}{lrr}
\caption*{
{\large Fremkomsmiddel}
} \\ 
\toprule
beskrivelse & N & prosent \\ 
\midrule\addlinespace[2.5pt]
Personbil/Varebil & 19 & 29.7 \\ 
Sykkel & 15 & 23.4 \\ 
Til fots & 10 & 15.6 \\ 
Elektrisk sparkesykkel & 4 & 6.2 \\ 
Motorsykkel & 4 & 6.2 \\ 
Ukjent & 4 & 6.2 \\ 
NA & 3 & 4.7 \\ 
ATV/Firhjuling & 2 & 3.1 \\ 
Buss & 1 & 1.6 \\ 
Lastebil & 1 & 1.6 \\ 
Trikk/tog/bane & 1 & 1.6 \\ 
Total & 64 & 100.0 \\ 
\bottomrule
\end{longtable*}
\endgroup

\subsection{Lettere skadd AIS 1 og 2}\label{lettere-skadd-ais-1-og-2}

Fordelingen av lettere skadde pasienter med AIS 1 eller 2 (N = 960)
etter kjønn, aldersgrupper og fremkomstmiddel er som følger:

\begingroup
\fontsize{12.0pt}{14.4pt}\selectfont
\begin{longtable*}{lrr}
\caption*{
{\large Kjønn fordeling}
} \\ 
\toprule
beskrivelse & N & prosent \\ 
\midrule\addlinespace[2.5pt]
Mann & 565 & 58.9 \\ 
Kvinne & 395 & 41.1 \\ 
Total & 960 & 100.0 \\ 
\bottomrule
\end{longtable*}
\endgroup

\begingroup
\fontsize{12.0pt}{14.4pt}\selectfont
\begin{longtable*}{lrr}
\caption*{
{\large Aldersgrupper}
} \\ 
\toprule
Alder & N & prosent \\ 
\midrule\addlinespace[2.5pt]
0-14 & 92 & 9.6 \\ 
15-24 & 162 & 16.9 \\ 
25-44 & 344 & 35.8 \\ 
45-64 & 253 & 26.4 \\ 
65-79 & 82 & 8.5 \\ 
80+ & 27 & 2.8 \\ 
Total & 960 & 100.0 \\ 
\bottomrule
\end{longtable*}
\endgroup

\begingroup
\fontsize{12.0pt}{14.4pt}\selectfont
\begin{longtable*}{lrr}
\caption*{
{\large Fremkomsmiddel}
} \\ 
\toprule
beskrivelse & N & prosent \\ 
\midrule\addlinespace[2.5pt]
Personbil/Varebil & 294 & 30.6 \\ 
Sykkel & 191 & 19.9 \\ 
Til fots & 147 & 15.3 \\ 
Ukjent & 69 & 7.2 \\ 
Elektrisk sparkesykkel & 61 & 6.4 \\ 
NA & 45 & 4.7 \\ 
Motorsykkel & 41 & 4.3 \\ 
Buss & 32 & 3.3 \\ 
Moped & 22 & 2.3 \\ 
El-sykkel & 21 & 2.2 \\ 
Annet spesifisert & 16 & 1.7 \\ 
Trikk/tog/bane & 6 & 0.6 \\ 
Sparkesykkel & 4 & 0.4 \\ 
Lastebil & 4 & 0.4 \\ 
Ski/Snøbrett mv. & 3 & 0.3 \\ 
Snøskuter & 3 & 0.3 \\ 
ATV/Firhjuling & 1 & 0.1 \\ 
Total & 960 & 100.0 \\ 
\bottomrule
\end{longtable*}
\endgroup




\end{document}
